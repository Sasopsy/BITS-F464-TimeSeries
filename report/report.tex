\documentclass[12pt,a4paper]{article}

% Encoding and language settings
\usepackage[utf8]{inputenc}
\usepackage[T1]{fontenc}
\usepackage[english]{babel}

% Math packages
\usepackage{amsmath}
\usepackage{amsfonts}
\usepackage{amssymb}

% For images
\usepackage{graphicx}

% Improving interface for defining floating objects such as figures and tables
\usepackage{float}

% Enhanced support for captions
\usepackage{caption}
\usepackage{subcaption}

% Margins and spacing
\usepackage[a4paper,margin=1in]{geometry}
\usepackage{setspace}
\doublespacing % or \onehalfspacing

% Hyperlinks in the document
\usepackage{hyperref}
\hypersetup{
    colorlinks=true,
    linkcolor=blue,
    filecolor=magenta,      
    urlcolor=cyan,
}

% Title, authors, and date configuration
\title{\textbf{Performance of Hypothesis Classes on Time Series Data}}
\author{
  Sasmit Datta\and
  Debangshu Chowdhary\and
  Netra Poonia
}
\date{}

\begin{document}

\maketitle

\section{Problem Statement}
This report aims to evaluate the predictive capabilities of three distinct models: a linear model, a decision tree, and a neural network, to determine which of them provides the most accurate and reliable forecasts for a given time series dataset. Each model comes with its own strengths and weaknesses concerning time series data - linear models are simple and interpretable but may fail to capture complex patterns; decision trees offer a more nuanced approach to non-linearity but may overfit; and neural networks offers a parametrized way to fit non-linear data but can suffer from high computational demands.
\subsection{Formulation}
For given time series data $\mathbf{x}_{T-k},\mathbf{x}_{T-k+1},\hdots,\mathbf{x}_{T}$ where $\mathbf{x}_t\in \mathbb{R}^n$ we fit a model 
\begin{equation}
f(\mathbf{x}_{T-k},\mathbf{x}_{T-k+1},...,\mathbf{x}_{T-1}) = \mathbf{x}_{T}^{(n)}
\end{equation}
where $\mathbf{x}_{t}^{(i)}\in \mathbb{R}$ is the $i$-th value of the vector $\mathbf{x}_{t}$. In essence what we are trying to do is given the multivariate time-series data points of the $k$ previous time steps, we try to predict the scaler value of the $n$-th index of data point of the next time-step.

\section{Methodology}

\subsection{Dataset}


% Rest of your document
\end{document}


